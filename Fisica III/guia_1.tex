\documentclass[leqno, 12pt, twoside, letterpaper]{book}
\usepackage{../packages/notas}
\usepackage{esvect}

\def\efield{\textbf{E}}
\def\edensity{\textbf{D}}
\def\vdiff{\textbf{dS}}

\def\lineInt#1#2#3{\int\displaylimits_{#2}^{#3} #1 \cdot d\textbf{l}}

\begin{document}


\section*{Ejercicio 8}

Queremos ver que el campo eléctrico va a mantener la simetría para los distintos casos

\subsection*{a) Simetria traslacional}

El objetivo es ver que  $\,\efield(\vv{r}) = \efield(\vv{r} + \vv{a})$, si expandimos la integral del campo tenemos

\[ \efield(\vv{r}) = k \int \lambda(\vv{v}) \dfrac{(\vv{r} - \vv{v})}{|\vv{r} - \vv{v}|^3} dV \]

\noindent si escribimos lo mismo para $\vv{r} + \vv{a}$ 


\[ \efield(\vv{r} + \vv{a}) = k \int \lambda(\vv{v}) \dfrac{(\vv{r} + \vv{a} - \vv{v})}{|\vv{r} + \vv{a} - \vv{v}|^3} dV \]

\noindent podemos tomar el cambio de variable $\vv{v}' = \vv{v} - \vv{a} $ con $dV' =  dV $ y obtener 


\[ \efield(\vv{r} + \vv{a}) = k \int \lambda(\vv{v}' + \vv{a}) \dfrac{(\vv{r} - \vv{v'})}{|\vv{r} - \vv{v'}|^3} dV' \]

\noindent Pero como además nos dijeron que la densidad de carga no cambia para las traslaciones $\lambda(\vv{v}') = \lambda(\vv{v}'+\vv{a})$

$$ \efield(\vv{r} + \vv{a}) = k \int \lambda(\vv{v}') \dfrac{(\vv{r} - \vv{v'})}{|\vv{r} - \vv{v'}|^3} dV' =  \efield(\vv{r})
$$

\subsection*{b) y c)} 

Salen haciendo lo mismo pero considerando las rotaciones y las reflexiones y considerando que $ R R^t \vv{v} = \vv{v}$

\section*{Ejercicio 9}

\textbf{a)} Queremos utilizar un argumento de simetría para ver la dirección del campo electrico en una linea de carga infinita.

Para esto vamos a ver dos cosas, una es que el campo electrico solo tiene parte vectorial radial y además solo depende del radio, es decir $\vv{\efield}(r\vv{e}_r + \phi\vv{e}_{\phi} + z\vv{e}_z ) = E(r) \vv{e}_r$ 

%TODO: argumentar bien la simetria

Nos piden hallar el campo usando gauss. Como solo tenemos una componente radial necesitamos una superficie de gauss que tenga simetría radial (horizontalmente) también. Podemos utilizar en este caso un cilindro.

Centrando un cilindro en el centro del eje de coordenadas de altura $L$ y radio $r$, podemos usar el teorema de Gauss y ver que 

\[ \int\displaylimits_{cilindro} \efield \cdot \vdiff = \dfrac{Q_{enc}}{\varepsilon_0}  \]

La carga encerrada es la de la longitud de la linea encerrada en el cilindro, en este caso también $L$, por la densidad de carga $\lambda$, $Q_{enc} = 2L\lambda/\varepsilon_0$

\hfill\\
\noindent Si separamos la integral en las integrales de cada tapa y la integral de los costados tenemos

\[ \int\displaylimits_{cilindro} \efield \cdot \vdiff = \int\displaylimits_{tapa\,\, superior} \efield \cdot \vv{e_z}dS  + \int\displaylimits_{costados} \efield \cdot \vv{e_r}dS  + \int\displaylimits_{tapa\,\, inferior} \efield \cdot (-\vv{e_z}dS )\]

Como $\efield(\vv{v}) = E(r) \vv{e_r} $, entonces $E(r) \vv{e_r}  \cdot \vv{e}_z dS = 0$ y las integrales de la tapa se anulan. Por otro lado $\efield$ es constante en los costados ya que el radio del cilindro es constante por lo que podemos sacarlo afuera de la integral, el producto interno además da $1$.

\[ \int\displaylimits_{cilindro} E(r) \vv{e_r} \cdot \vv{e_r}dS =   E(r)  \int\displaylimits_{costados}dS \]

Como sabemos que el área del costado es $2\pi r L$ tenemos que

\[ E(r) 2\pi  L = \dfrac{Q_{enc}}{\varepsilon_0} = \dfrac{L}{\varepsilon_0} \]

entonces

\[ E(r) = \dfrac{\lambda L}{2\pi r L \varepsilon_0}\]

y como $\efield(\vv{v}) = E(r)_r \vv{e}_r$

\[ \efield(v) = \dfrac{\lambda \vv{e}_r}{2\pi  \varepsilon_0r} \]


Ahora nos piden calcular el potencial $V$ para esto tenemos que integrar desde un punto fuente a un punto $P$ que es donde queremos evaluar el potencial, es decir.

\[ V(\vv{P}) = -\lineInt{\efield}{\vv{P}}{\mathfrak{F}}  \] 

Como no nos importa la altura ni el angulo de rotación dado $P = (r, \phi, z)$ podemos tomar la curva $\sigma(t) = (t, \phi, z)$ con $t \in [1, r]$ (ahora vamos a ver porque el punto de referencia es $1$).

Entonces $\sigma(t)' = (1, 0, 0)$ y tenemos que


$$ V(\vv{P}) = -\lineInt{\efield}{\vv{P}}{\mathfrak{F}} = - \int\displaylimits_{1}^{r} \efield(\sigma(t)) \, dt  
=  - \int\displaylimits_{1}^{r} \dfrac{\lambda L}{2\pi t L \varepsilon_0} \, dt 
=  - \dfrac{\lambda }{2\pi  \varepsilon_0} \int\displaylimits_{1}^{r} \dfrac{1}{t} \, dt = $$
$$ - \dfrac{\lambda }{2\pi  \varepsilon_0} \ln(t) \Big\rvert_{r}^{1} = - \dfrac{\lambda }{2\pi  \varepsilon_0} \ln(r) 
$$ 

Confirmemos que esto es correcto, tomemos el gradiente de $V$ y veamos que da $\efield$

$$ - \nabla V = - \Big(\dfrac{dV}{dr} \vv{e}_r + \dfrac{dV}{d\phi} \vv{e}_{\phi} + \dfrac{dV}{dz} \vv{e}_z\Big) = - - \dfrac{\lambda}{2\pi \varepsilon_0} \dfrac{1}{r} \vv{e}_r =  \dfrac{\lambda}{2\pi \varepsilon_0 r} \vv{e}_r $$

\hfill\\\hfill\\
\textbf{b)} Usando un análisis análogo al punto a) vemos que el campo apunta radialmente hacia afuera es decir tenemos que $\efield(\vv{v}) = E(r) \vv{e}_r$ (Esto vale tanto como para adentro del volumen como afuera de este). También igual que en el anterior vamos a elegir como superficie gaussiana un cilindro de altura $L$, esta vez tenemos que analizar dos casos, en el exterior y en el interior. Además notemos que el radio del cilindro cargado es $R$ y el de la superficie gaussiana es $r$. 
\hfill\\

En el exterior tenemos de vuelta que  

\[ \int\displaylimits_{cilindro} \efield \cdot \vdiff = \dfrac{Q_{enc}}{\varepsilon_0} = \dfrac{\lambda\pi R^2 L}{\varepsilon_0} \]

Como $E(r)$ es constante si no cambia el radio

\[ \int\displaylimits_{costados} E(r) \, dS = E(r) \int\displaylimits_{costados}  \, dS = E(r) 2 \pi r L\]

\[ E(r) 2 \pi r L = \dfrac{\lambda\pi R^2 L}{\varepsilon_0} \Rightarrow  E(r) \vv{e}_r = \dfrac{\lambda R^2 }{2 r \varepsilon_0} \vv{e}_r\]

En el caso interior vamos a tener casi lo mismo, solo que en vez de $R$ la carga contenida va a depender de $r$ y por lo tanto 

\[ E(r) \vv{e}_r = \dfrac{\lambda r }{2 \varepsilon_0} \vv{e}_r\]

Juntando tenemos

$$ \efield(r) = \twopartdef{\dfrac{\lambda r }{2 \varepsilon_0} \vv{e}_r}{ r < R }{\dfrac{\lambda R^2 }{2 r \varepsilon_0} \vv{e}_r}{\textrm{si } r > R} $$


\section*{Ejercicio 11} 

Queremos ver la carga contenida en el cilindro, usando Gauss y usando que el campo electrico se orienta hacia la tierra, por lo cual la integral de gauss sobre los costados es 0 y solo integramos las tapas ($T1$ la tapa superior y $T2$ la inferior), tenemos que

\[ \int \efield \cdot \vdiff = \dfrac{20 \, V}{m} \int\displaylimits_{T1} dS +  \dfrac{300 \,V}{m}\int\displaylimits_{T2} dS = \dfrac{320 \, V}{m} \pi r^2 \]

entonces
 
$$  \dfrac{320 \, V \pi r^2 }{m \, \varepsilon_0} = 2.72nC \pi r^2 =  Q_{\textrm{enc}} $$
\end{document}
