\documentclass[leqno, 12pt, twoside, letterpaper]{book}
\usepackage{packages/notas}
\usepackage{esvect}

\def\efield{\textbf{E}}

\begin{document}


\section*{Ejercicio 8}

Queremos ver que el campo eléctrico va a mantener la simetría para los distintos casos

\subsection*{a) Simetria traslacional}

El objetivo es ver que  $\,\efield(\vv{r}) = \efield(\vv{r} + \vv{a})$, si expandimos la integral del campo tenemos

\[ \efield(\vv{r}) = k \int \lambda(\vv{v}) \dfrac{(\vv{r} - \vv{v})}{|\vv{r} - \vv{v}|^3} dV \]

\noindent si escribimos lo mismo para $\vv{r} + \vv{a}$ 


\[ \efield(\vv{r} + \vv{a}) = k \int \lambda(\vv{v}) \dfrac{(\vv{r} + \vv{a} - \vv{v})}{|\vv{r} + \vv{a} - \vv{v}|^3} dV \]

\noindent podemos tomar el cambio de variable $\vv{v}' = \vv{v} - \vv{a} $ con $dV' =  dV $ y obtener 


\[ \efield(\vv{r} + \vv{a}) = k \int \lambda(\vv{v}' + \vv{a}) \dfrac{(\vv{r} - \vv{v'})}{|\vv{r} - \vv{v'}|^3} dV' \]

\noindent Pero como además nos dijeron que la densidad de carga no cambia para las traslaciones $\lambda(\vv{v}') = \lambda(\vv{v}'+\vv{a})$

$$ \efield(\vv{r} + \vv{a}) = k \int \lambda(\vv{v}') \dfrac{(\vv{r} - \vv{v'})}{|\vv{r} - \vv{v'}|^3} dV' =  \efield(\vv{r})
$$

\end{document}
