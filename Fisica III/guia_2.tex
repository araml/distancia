\documentclass[leqno, 12pt, twoside, letterpaper]{book}
\usepackage{../packages/notas}
\usepackage{esvect}
\usepackage[bookmarks=true]{hyperref}
\usepackage[numbered]{bookmark}
\hypersetup{ pdfpagemode = UseOutlines }

\def\ejercicio#1{\pdfbookmark[section]{Ejercicio #1}{ej#1}\section*{Ejercicio #1}}
\def\itemEj#1{\pdfbookmark[subsection]{#1}{sub#1}\noindent\textbf{#1)}}

\begin{document}

\ejercicio{1}
\itemEj{i} Como siempre en el caso electrostatico las cargas en los conductores
    se van a distribuir sobre las superficies de los mismos, en este caso en el
    conductor interno vamos a tener una carga externa de $1nC$, el conductor de
    más afuera va a intentar compensar esta carga con una carga inducida en
    superficie interna de $-1nC$, para ver esto solo basta considerar una superficie
    que recorra internamente el conductor de más afuera, como el campo en este
    tiene que ser $0$ la carga encerrada también, por lo que hay una carga que
    cancela el $1nC$ del conductor de más adentro.
    Externamente teniamos la carga de $2nC$ menos la carga que
    contrarresto la carga interna, pero esta es $-1nC$ osea que en la superficie
    externa vamos a ver $3nC$ de carga.\\

\itemEj{ii} Si ambos conductores se tocan entonces vamos a pasar a tener un
    conductor grande con un hueco interno, de vuelta la carga tiende a la
    superficie externa por lo cual el borde interno no va a tener carga y vamos
    a tener de vuelta $3nC$ de carga. \\

\itemEj{iii} \hfill \\

\ejercicio{3}

\end{document}
