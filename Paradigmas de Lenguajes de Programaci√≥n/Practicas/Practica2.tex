\chapter{Practica 2}


\exercise{Ejercicio 3} \\

\inciso{a)} Marco los subtérminos de cada termino con corchetes y el árbol muestra como se
separan, los subtérminos que quiero encontrar están en rojo. \\

\Tree[.{$\lambda x$: Nat. $\underbrace{succ((\lambda x: Nat. \, x) \, x)}$}
          [.{$succ(\underbrace{(\lambda x: Nat. \, x) \, x)})$}
            [.{$\underbrace{(\lambda x: Nat. \, x)} \, \underbrace{x}$}
              [.{$\lambda x: Nat. \, \underbrace{x}$}
                [.{\color{red}{$x$}}
                ]
              ]
              [.{\color{red}{$x$}}
              ]
            ]
          ]
     ]

\inciso{b)} No?

\inciso{c)} Aunque la gramática no lo hace, siempre asociamos a izquierda, es
decir tenemos los subtérminos

\[ \underbrace{u \, x} \underbrace{(y\, z)} \]

\noindent por lo cual $x \, (y \, z)$ no puede ser un subtérmino de la
expresión. \\


\exercise{Ejercicio 4} \\

\inciso{I.}

\inciso{II.}

\inciso{III.}

\inciso{IV.} Como asocia a izquierda podemos ver que esto es subexpresion de
$b$.\\


\exercise{Ejercicio 5} \\

\inciso{a)}

\begin{center}
    \begin{prooftree}
        \TTTrue{\juicioVacio true : Bool}
        \TTZero{\juicioVacio 0 : Nat}
        \TTZero{\juicioVacio 0 : Nat}
        \infer1[ \TSucc ]{\juicioVacio succ(0) : Nat}
        \infer3[ \TIf ]{ \juicioVacio \ITE{true}{0}{succ(0)} : Nat}
    \end{prooftree}
\end{center}

\inciso{b)} Tomando $\Gamma = \{x : Nat, y : Bool\}$

\begin{center}
\scalebox{0.65}{
    \begin{prooftree}
        \TTTrue{\juicio true : Bool}
        \TTFalse{\juicio false : Bool}
        \TTVar{\juicioU{z: Bool} z : Bool}
        \infer1[ \TAbs ]{\juicio \labs{z}{Bool}{z}{\sigma = Bool}{Bool}}
        \TTTrue{\juicio true : Bool}
        \infer2[ \TApp ]{\juicio (\labsApp{z}{Bool}{z}) \, true : Bool}
        \infer3[ \TIf ]{ \juicio \ITE{true}{false}{(\labsApp{z}{Bool}{z}) \, true} :
                           Bool}
    \end{prooftree}
}
\end{center}

\inciso{c)}

\begin{center}
\scalebox{1}{
    \begin{prooftree}
        \hypo[]{\text{Error, el tipo de una lambda no puede ser Bool}}
        \infer1[ (\text{No Tipa}) ]{\juicioVacio \labsApp{x}{Bool}{x} : Bool }
        \infer0[]{}
        \infer0[]{}
        \infer3[ \TIf ]{ \juicioVacio \ITE{\labsApp{x}{Bool}{x}}{0}{succ(0)} : Nat}

    \end{prooftree}
}
\end{center}

\inciso{d)}  Tomando $\Gamma = \{x : Bool \rightarrow  Nat, y : Bool\}$

\begin{center}
\scalebox{1}{
    \begin{prooftree}
        \hypo[ ]{ \sigma = Bool, \tau = Nat}
        \infer1[ \TVar ]{ x : \sigma \rightarrow \tau}
        \hypo[]{ \sigma = Bool }
        \infer1[ \TVar ]{\juicio y : \sigma }
        \infer2[ \TApp ]{ \juicio x y: Nat}

    \end{prooftree}
}
\end{center}

\exercise{Ejercicio 6}\\

\noindent Medio que se ven a ojo, no se si habría que hacer un juicio de tipado..

\inciso{a)} Tipa si $\sigma = Nat$.\\
\inciso{b)} Lo mismo con $\sigma = Bool$.\\
\inciso{c)} De vuelta $\sigma = Nat$. \\


\exercise{Ejercicio 7}

\begin{center}
\scalebox{1}{
    \begin{prooftree}
        \hypo[]{ \sigma = Nat, \tau = Bool}
        \infer1[ \TVar ] {\juicioSet{x : \sigma} x : Nat}
        \infer1[ \TSucc ]{ \juicioSet{x : \sigma} succ(x) : Nat }
        \infer1[ \TIsZero ]{\juicioSet{x : \sigma} isZero(succ(x)) : \tau}
    \end{prooftree}
}
\end{center}

\exercise{Ejercicio 10}
\inciso{a)} \[ (\labsApp{y}{\sigma}{x (\labsApp{x}{\tau}{x})})\{ x \leftarrow
(\labsApp{y}{\rho}{\underbrace{x}_{\text{libre}} y}) \}  = \]

\[ \labsApp{y}{\sigma}{x (\labsApp{x}{\tau}{x})}\{ x \leftarrow
(\labsApp{y}{\rho}{\underbrace{x}_{\text{libre}} y}) \}  = \]

\noindent Viendo solo el cuerpo de la función lambda tenemos

\[ x (\labsApp{x}{\tau}{x})\{ x \leftarrow
(\labsApp{y}{\rho}{\underbrace{x}_{\text{libre}} y}) \}  = \]


\[ x\{ x \leftarrow (\labsApp{y}{\rho}{\underbrace{x}_{\text{libre}} y}) \}
\quad (\labsApp{x}{\tau}{x}) \{ x \leftarrow (\labsApp{y}{\rho}{\underbrace{x}_{\text{libre}} y}) \}
= \]

\noindent Cambiando el nombre de la variable ligada

\[ (\labsApp{y}{\rho}{x \, y})  \quad (\labsApp{w}{\tau}{w}) \{ x \leftarrow (\labsApp{y}{\rho}{\underbrace{x}_{\text{libre}} y}) \}
= \]

\[ (\labsApp{y}{\rho}{x \, y})  \quad (\labsApp{w}{\tau}{w}) \]

\noindent y uniendo todo tenemos

\begin{exbox}
    \[ \labsApp{y}{\sigma}{ (\labsApp{y}{\rho}{x \, y})  \,
    (\labsApp{w}{\tau}{w}) } \]
\end{exbox}



\exercise{Ejercicio 17}
\inciso{a)}


\[
    \begin{prooftree}
        \infer0[ T-Vacio ]{\juicioVacio []_{\sigma}: [\sigma]}
    \end{prooftree}
    \qquad
    \begin{prooftree}
        \hypo{\juicio M : \sigma}
        \hypo{\juicio N : [\sigma]}
        \infer2[T-Encolar ]{\juicio M::N : [\sigma]}
    \end{prooftree}
\]
\\
\[
    \begin{prooftree}
        \hypo{\juicio M : [\sigma]}
        \hypo{\juicio N : \tau}
        \hypo{\juicioUnion{\pset{h:\sigma, t:[\sigma]}} O : \tau}
        \infer3[(T-CaseOf)]{\juicio Case \, M \, of \{ [] \rightsquigarrow N |
        h::t \rightsquigarrow O \}: [\tau]}
    \end{prooftree}
\]
\\

\[
    \begin{prooftree}
        \hypo{\juicio M : [\sigma]}
        \hypo{\juicio N : \tau}
        \hypo{\juicioUnion{\pset{h:\sigma, r:tau}} O : \tau}
        \infer3[(T-foldR)]{\juicio foldr \, M \, base \rightsquigarrow N; rec(h,
        r) \rightsquigarrow O : \tau}
    \end{prooftree}
\]
\\

\inciso{d)} Valores $ V:= []_{\sigma}, M::N$
\hfill\\

\inciso{e)}
\[
    \begin{prooftree}
        \infer0[(E-CaseOfEmpty)]{\juicio Case \, []_{\sigma} \, of \{ [] \rightsquigarrow N |
        h::t \rightsquigarrow O \} \rightarrow N}
    \end{prooftree}
\]
\\
\[
    \begin{prooftree}
        \hypo{\juicioVacio M : \sigma}
        \hypo{\juicioVacio N : [\tau]}
        \infer2[(E-CaseOf)]{\juicio Case \, M::N \, of \{ [] \rightsquigarrow N |
        h::t \rightsquigarrow O \} \rightarrow O\pset{h \leftarrow M, t
        \leftarrow N}}
    \end{prooftree}
\]
\\
\[
    \begin{prooftree}
        \infer0[(E-foldRVacio)]{\juicio foldr \, []_{\sigma} \, base \rightsquigarrow N; rec(h,
        r) \rightsquigarrow O \rightarrow N}
    \end{prooftree}
\]
\\
\scalebox{0.8}{
    \begin{prooftree}
        \infer0[(E-foldR)]{\juicio foldr \, M::N \, base \rightsquigarrow N; rec(h,
        r) \rightsquigarrow O \rightarrow O\pset{h \leftarrow M, rec \leftarrow
        foldr \, N \, base \rightsquigarrow N; rec(h,
        r) \rightsquigarrow O }}
    \end{prooftree}
}
