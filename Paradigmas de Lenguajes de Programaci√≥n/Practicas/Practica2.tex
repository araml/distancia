\chapter{Practica 2}


\exercise{Ejercicio 3} \\

\inciso{a)} Marco los subtérminos de cada termino con corchetes y el árbol muestra como se
separan, los subtérminos que quiero encontrar están en rojo. \\

\Tree[.{$\lambda x$: Nat. $\underbrace{succ((\lambda x: Nat. \, x) \, x)}$}
          [.{$succ(\underbrace{(\lambda x: Nat. \, x) \, x)})$}
            [.{$\underbrace{(\lambda x: Nat. \, x)} \, \underbrace{x}$}
              [.{$\lambda x: Nat. \, \underbrace{x}$}
                [.{\color{red}{$x$}}
                ]
              ]
              [.{\color{red}{$x$}}
              ]
            ]
          ]
     ]

\inciso{b)} No?

\inciso{c)} Aunque la gramática no lo hace, siempre asociamos a izquierda, es
decir tenemos los subtérminos

\[ \underbrace{u \, x} \underbrace{(y\, z)} \]

\noindent por lo cual $x \, (y \, z)$ no puede ser un subtérmino de la
expresión. \\


\exercise{Ejercicio 4} \\

\inciso{I.}
