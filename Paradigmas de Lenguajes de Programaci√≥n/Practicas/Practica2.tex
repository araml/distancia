\chapter{Practica 2}


\exercise{Ejercicio 3} \\

\inciso{a)} Marco los subtérminos de cada termino con corchetes y el árbol muestra como se
separan, los subtérminos que quiero encontrar están en rojo. \\

\Tree[.{$\lambda x$: Nat. $\underbrace{succ((\lambda x: Nat. \, x) \, x)}$}
          [.{$succ(\underbrace{(\lambda x: Nat. \, x) \, x)})$}
            [.{$\underbrace{(\lambda x: Nat. \, x)} \, \underbrace{x}$}
              [.{$\lambda x: Nat. \, \underbrace{x}$}
                [.{\color{red}{$x$}}
                ]
              ]
              [.{\color{red}{$x$}}
              ]
            ]
          ]
     ]

\inciso{b)} No?

\inciso{c)} Aunque la gramática no lo hace, siempre asociamos a izquierda, es
decir tenemos los subtérminos

\[ \underbrace{u \, x} \underbrace{(y\, z)} \]

\noindent por lo cual $x \, (y \, z)$ no puede ser un subtérmino de la
expresión. \\


\exercise{Ejercicio 4} \\

\inciso{I.}

\inciso{II.}

\inciso{III.}

\inciso{IV.} Como asocia a izquierda podemos ver que esto es subexpresion de
$b$.\\


\exercise{Ejercicio 5} \\

\inciso{a)}

\begin{center}
    \begin{prooftree}
        \TTTrue{\juicioVacio true : Bool}
        \TTZero{\juicioVacio 0 : Nat}
        \TTZero{\juicioVacio 0 : Nat}
        \infer1[ \TSucc ]{\juicioVacio succ(0) : Nat}
        \infer3[ \TIf ]{ \juicioVacio \ITE{true}{0}{succ(0)} : Nat}
    \end{prooftree}
\end{center}

\inciso{b)} Tomando $\Gamma = \{x : Nat, y : Bool\}$

\begin{center}
\scalebox{0.65}{
    \begin{prooftree}
        \TTTrue{\juicio true : Bool}
        \TTFalse{\juicio false : Bool}
        \TTVar{\juicioU{z: Bool} z : Bool}
        \infer1[ \TAbs ]{\juicio \labs{z}{Bool}{z}{\sigma = Bool}{Bool}}
        \TTTrue{\juicio true : Bool}
        \infer2[ \TApp ]{\juicio (\labsApp{z}{Bool}{z}) \, true : Bool}
        \infer3[ \TIf ]{ \juicio \ITE{true}{false}{(\labsApp{z}{Bool}{z}) \, true} :
                           Bool}
    \end{prooftree}
}
\end{center}

\inciso{c)}

\begin{center}
\scalebox{1}{
    \begin{prooftree}
        \hypo[]{\text{Error, el tipo de una lambda no puede ser Bool}}
        \infer1[ (\text{No Tipa}) ]{\juicioVacio \labsApp{x}{Bool}{x} : Bool }
        \infer0[]{}
        \infer0[]{}
        \infer3[ \TIf ]{ \juicioVacio \ITE{\labsApp{x}{Bool}{x}}{0}{succ(0)} : Nat}

    \end{prooftree}
}
\end{center}

\inciso{d)}  Tomando $\Gamma = \{x : Bool \rightarrow  Nat, y : Bool\}$

\begin{center}
\scalebox{1}{
    \begin{prooftree}
        \hypo[ ]{ \sigma = Bool, \tau = Nat}
        \infer1[ \TVar ]{ x : \sigma \rightarrow \tau}
        \hypo[]{ \sigma = Bool }
        \infer1[ \TVar ]{\juicio y : \sigma }
        \infer2[ \TApp ]{ \juicio x y: Nat}

    \end{prooftree}
}
\end{center}

\exercise{Ejercicio 6}
