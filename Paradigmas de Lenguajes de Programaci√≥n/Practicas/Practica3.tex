\chapter{Practica Nº 3 - Subtipado}

\exercise{Ejercicio 7} \\

\inciso{a)}

\scalebox{0.85}{
    \begin{prooftree}
        \hypo{\checkmark}
        \infer1[ \TVar ]{\juicioSet{x : Bool, y : Nat} y : Nat }
        \infer1[\TSucc ]{ \juicioSet{x : Bool, y : Nat} suc(y) : Nat }
        \infer1[ \TAbs ]{\juicioSet{x : Bool} \labsApp{y}{Nat}{suc(y)} : Nat \to Nat }
        \hypo{\checkmark}
        \infer1[ \TVar]{\juicioSet{x : Bool} x : Bool}
        \hypo{ Bool <: Nat }
        \infer2[ (S-Sub) ]{\juicioSet{x : Bool} x : Nat }
        \infer2[ \TApp ]{\juicioSet{x : Bool} \labsApp{y}{Nat}{suc(y)} x : Nat}
        \infer1[ \TAbs ]{ \juicioVacio
        \labsApp{x}{Bool}{(\labsApp{y}{Nat}{suc(y)}) x} : Bool \to Nat }
    \end{prooftree}
}


\hfill\\
\exercise{Ejercicio 11} \\

\inciso{a)}

\scalebox{0.8} {
    \begin{prooftree}
        \hypo{\checkmark, \alpha = \pset{x : Int}}
        \infer1[\TVar]{\juicio c : Comp_{\alpha} }
        \hypo{\checkmark}
        \infer1[(AxiomaXY)]{ \juicio  \pset{x = 1, y = 2} :  \pset{x : Int, y : Int} }
        \infer1[(S - RcdWidth)]{\juicio  \pset{x = 1, y = 2} :  \pset{x : Int} }
        \infer1[]{\juicio  \pset{x = 1, y = 2} : \alpha }
        \hypo{\checkmark}
        \infer1[\TZero]{\juicio 0 : Int}
        \infer1[(T - Rcd)]{\juicio \pset{x = 0} : \pset{x : Int}}
        \infer1{\juicio \pset{x = 0} : \alpha}
        \infer3[(T - Comp)]{\Gamma = \juicioSet{c : Comp_{\pset{x : Int}}} \text{mejorSegún}(c, \pset{x =
        1, y = 2}, \pset{x = 0}) : Bool}
        \infer1[\TAbs]{\juicioVacio \labsApp{c}{Comp_{\pset{x : Int}}}{ \text{mejorSegún}(c, \pset{x =
        1, y = 2}, \pset{x = 0})} : \sigma \to Bool}
    \end{prooftree}
}

\inciso{b)}

Queremos ver la regla de subtipado de $Comp_{\sigma}$. Comp compara dos valores
del mismo tipo, por lo cual si tuviésemos $Comp_{\sigma} <: Comp_{\tau}$ si
$\sigma <: \tau$ esto haría que no podamos comparar todo el dominio, por lo
tanto tenemos que  $Comp_{\sigma} <: Comp_{\tau}$ si $\tau <: \sigma$, es decir
la relación de subtipado es contravariante.

\inciso{c)}

\hfill\\
\exercise{Ejercicio 12} \\

\inciso{a)}
\begin{solucion*}
\begin{center}
\scalebox{0.45}{
    \begin{prooftree}
        \infer0[ (T-Vaca) ]{\juicio Clarabelle : Vaca}
        \infer0[ (S-Vaca) ]{Vaca <: Animal}
        \infer2[ (T-Subs) ]{\juicio Clarabelle : Animal}
        \infer0[ (S-Vaca)] {\juicio Clarabelle : Vaca}
        \infer0[ (S-VacaLeon) ]{Vaca <: AlimentoPara(Leon)}
        \infer2[ (T-Subs) ]{\juicio Clarabelle : AlimentoPara(Leon)}
        \infer0[ (S-Leon) ]{Leon <: Animal}
        \infer1[ (S-Alim) ]{AlimentoPara(Leon) <: AlimentoPara(Animal)}
        \infer2[ (T-Subs)]{\juicio Clarabelle : AlimentoPara(Animal)}
        \infer2[ (T-Comer) ]{ \juicio comer(Clarabelle, Clarabelle) : Animal}
    \end{prooftree}
}
\end{center}

\hfill\\


\inciso{b)} El problema es que S-Alim permite reemplazar animal a la derecha por
cualquier tipo que sea subtipo de animal, esto debería estar al revez, osea ser
covariante, ya que queremos que si tenemos una vaca o león a la derecha su
alimento se pueda reemplazar por el de animal, pero no deberíamos poder
reemplazar el alimento de un animal por el de una vaca o león. Entonces la regla
quedaría:

\[
\begin{prooftree}
    \hypo{\sigma' <: \sigma}
    \infer1[ (S-Alim) ]{AlimentoPara(\sigma) <: AlimentoPara(\sigma')}
\end{prooftree}
\]
\end{solucion*}
