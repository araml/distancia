\chapter{Practica Nº 4 - Subtipado}

\exercise{Ejercicio 12} \\

\inciso{a)}
\begin{solucion*}
\begin{center}
\scalebox{0.45}{
    \begin{prooftree}
        \infer0[ (T-Vaca) ]{\juicio Clarabelle : Vaca}
        \infer0[ (S-Vaca) ]{Vaca <: Animal}
        \infer2[ (T-Subs) ]{\juicio Clarabelle : Animal}
        \infer0[ (S-Vaca)] {\juicio Clarabelle : Vaca}
        \infer0[ (S-VacaLeon) ]{Vaca <: AlimentoPara(Leon)}
        \infer2[ (T-Subs) ]{\juicio Clarabelle : AlimentoPara(Leon)}
        \infer0[ (S-Leon) ]{Leon <: Animal}
        \infer1[ (S-Alim) ]{AlimentoPara(Leon) <: AlimentoPara(Animal)}
        \infer2[ (T-Subs)]{\juicio Clarabelle : AlimentoPara(Animal)}
        \infer2[ (T-Comer) ]{ \juicio comer(Clarabelle, Clarabelle) : Animal}
    \end{prooftree}
}
\end{center}

\hfill\\


\inciso{b)} El problema es que S-Alim permite reemplazar animal a la derecha por
cualquier tipo que sea subtipo de animal, esto debería estar al revez, osea ser
covariante, ya que queremos que si tenemos una vaca o león a la derecha su
alimento se pueda reemplazar por el de animal, pero no deberíamos poder
reemplazar el alimento de un animal por el de una vaca o león. Entonces la regla
quedaría:

\[
\begin{prooftree}
    \hypo{\sigma' <: \sigma}
    \infer1[ (S-Alim) ]{AlimentoPara(\sigma) <: AlimentoPara(\sigma')}
\end{prooftree}
\]
\end{solucion*}
