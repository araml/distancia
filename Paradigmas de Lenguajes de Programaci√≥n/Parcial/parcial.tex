\documentclass[leqno, 12pt, twoside, letterpaper]{book}
\usepackage{../packages/notas}
\usepackage{../packages/lambda}
\usepackage{color}
\usepackage{hyperref}
\usepackage{qtree}
\usepackage{minted}

\hypersetup{
    colorlinks=true,
    linkcolor=blue,
    urlcolor=red,
    linktoc=all,
    hypertexnames=false,
    linktocpage=true,
}
\usepackage{graphicx}

%\setlength\parindent{0pt}
\date{\vspace{-5ex}}

\def\checkmark{\tikz\fill[scale=0.4](0,.35) -- (.25,0) -- (1,.7) -- (.25,.15) -- cycle;}

\title{Autoevaluación 2do cuatrimestre 2020}
\makeindex

\begin{document}
\maketitle
\tableofcontents

\newpage

\exercise{Ejercicio 1 - Programación Funcional}\\


\inciso{a)}
\begin{minted}{Haskell}
ejGrafo :: Grafo
ejGrafo = G [1..9] (\x -> if mod x 2 == 0 then [1..9] else [])
\end{minted}

\inciso{b)}
\begin{minted}{Haskell}
nodoReflex :: Grafo -> Nodo -> Bool
nodoReflex G ns f n = n elem (f n)
\end{minted}

\inciso{e)}
\begin{minted}{Haskell}

ocurrencias :: Int -> [Int] -> Int
ocurrencias n ls = len (filter (\x -> x != n) ls)

sinNodosRepetidos :: [Int] -> Bool
sinNodosRepetidos ls = foldr (\x r ->
                              if (ocurrencias x ls) != 1 then
                                  false
                              else
                                true && r) true ls

hamiltoniano :: Grafo -> Bool
hamiltoniano G ns f = len (filter sinNodosRepetidos
                                  (caminosDeLong  (g n f) (len ns - 1))
\end{minted}

\inciso{f)}
\begin{minted}{Haskell}
tieneCiclo :: Grafo -> Bool
tieneCiclo G ns f = length (caminosDeLong (G ns f) (len ns + 1)) > 0
\end{minted}

\newpage

\exercise{Ejercicio 2 - Cálculo Lambda Tipado}\\

\inciso{a)}

\[
\begin{prooftree}
    \hypo{\juicio N : \sigma}
    \infer1[ (T - MatInit) ]{ \juicio MAT(N) : MatInf_{\sigma}}
\end{prooftree}
\]

\hfill

\[
\begin{prooftree}
    \hypo{\juicio M : MatInf_{\sigma} \quad  \juicio N, O: Nat}
    \infer1[ (T - MatObs) ]{ \juicio M[N][P] : \sigma}
\end{prooftree}
\]

\hfill

\[
\begin{prooftree}
    \hypo{\juicio M : MatInf_{\sigma} \quad \juicio P : \sigma \quad \juicio N, O: Nat}
    \infer1[ (T - MatIns) ]{ \juicio M[N][P] \leftarrow P : MatInf_{\sigma} }
\end{prooftree}
\]


\hfill

\[
\begin{prooftree}
    \hypo{\juicio M : MatInf_{\sigma} \quad \juicio N : \sigma \rightarrow \tau}
    \infer1[ (T - MatMap) ]{ \juicio map(M, N) : MatInf_{\tau}}
\end{prooftree}
\]

\hfill\\


\inciso{b)} Tomando $\Gamma = \pset{i : Nat, m : MatInf_{Nat}}$

\begin{center}
\scalebox{0.9}{
    \begin{prooftree}
	\hypo[]{\text{Sigue en *}}
	\infer1[ (T - Ins) ]{\juicio m[i][i] \leftarrow 0 : MatInf_{\sigma}}
	\hypo[]{ \text{Nota: aca llegue a que el } \sigma \text{ inicial es } Nat. }
	\infer1[]{\juicioU{x : Nat} isZero(x) : Bool}
        \infer1[ (T - Abs) ]{\juicio \labsApp{x}{Nat}{isZero(x))} : \sigma \rightarrow Bool}
        \infer2[ (T - Map) ]{ \juicio map((m[i][i] \leftarrow 0), \labsApp{x}{Nat}{isZero(x))} : MatInf_{Bool}}
    \end{prooftree}
}
\end{center}

Reescribiendo con el nuevo sigma

\begin{center}
\scalebox{0.9}{
    \begin{prooftree}
	\hypo[]{\text{Sigue en *}}
	\infer1[ (T - Ins) ]{\juicio m[i][i] \leftarrow 0 : MatInf_{Nat}}	
	\hypo[]{ \checkmark }
	\infer1[ (T - Var) ]{\juicioU{x : Nat} x : Nat}
	\infer1[ (T - IsZero) ]{\juicioU{x : Nat} isZero(x) : Bool}
        \infer1[ (T - Abs) ]{\juicio \labsApp{x}{Nat}{isZero(x))} : Nat \rightarrow Bool}
        \infer2[ (T - Map) ]{ \juicio map((m[i][i] \leftarrow 0), \labsApp{x}{Nat}{isZero(x))} : MatInf_{Bool}}
    \end{prooftree}
}
\end{center}

\newpage

\noindent Sigo la parte del $(*)$

\begin{center}
\scalebox{0.9}{
    \begin{prooftree}
    	\hypo[]{\checkmark}
	\infer1[(T - Var)]{\juicio m : MatInf_{Nat}}
	\hypo[]{\checkmark}
	\infer1[ (T - Zero)]{\juicio 0 : Nat}
	\hypo[]{\checkmark}
	\infer1[(T - Var)]{\juicio i : Nat}
	\hypo[]{\checkmark}
	\infer1[(T - Var)]{\juicio i : Nat}
	\infer4[ (T - Ins) ]{\juicio m[i][i] \leftarrow 0 : MatInf_{Nat}}	
	\infer1[]{*}
    \end{prooftree}
}
\end{center}


\noindent Y la demostracion entera, para fines de completitud:

\begin{center}
\scalebox{0.65}{
    \begin{prooftree}
    	\hypo[]{\checkmark}
	\infer1[(T - Var)]{\juicio m : MatInf_{Nat}}
	\hypo[]{\checkmark}
	\infer1[ (T - Zero)]{\juicio 0 : Nat}
	\hypo[]{\checkmark}
	\infer1[(T - Var)]{\juicio i : Nat}
	\hypo[]{\checkmark}
	\infer1[(T - Var)]{\juicio i : Nat}
	\infer4[ (T - Ins) ]{\juicio m[i][i] \leftarrow 0 : MatInf_{Nat}}	
	\hypo[]{ \checkmark }
	\infer1[ (T - Var) ]{\juicioU{x : Nat} x : Nat}
	\infer1[ (T - IsZero) ]{\juicioU{x : Nat} isZero(x) : Bool}
        \infer1[ (T - Abs) ]{\juicio \labsApp{x}{Nat}{isZero(x))} : Nat \rightarrow Bool}
        \infer2[ (T - Map) ]{ \juicio map((m[i][i] \leftarrow 0), \labsApp{x}{Nat}{isZero(x))} : MatInf_{Bool}}
    \end{prooftree}
}
\end{center}

\inciso{c)} En principio solo hay que agregar el valor $MatInf_{\sigma}$, para el resto nada ya que son valores que ya existen (inserción, observar). \\

\newpage

\inciso{d)} Como nos piden solo reducir en el caso de observar vamos a concentrar las reglas de evaluación ahí.

\begin{center}
\scalebox{0.95}{
\begin{prooftree}
    \hypo[]{ N \rightarrow N'}
    \infer1[ (E - MatObsCol) ]{ M[N][N_2] \rightarrow M[N'][N_2]}
\end{prooftree}
\qquad
\begin{prooftree}
    \hypo[]{ N \rightarrow N'}
    \infer1[ (E - MatObsRow) ]{ M[V][N] \rightarrow M[V][N']}
\end{prooftree}
}
\end{center}

\begin{center}
\scalebox{1}{
\begin{prooftree}
    \hypo[]{ N \rightarrow N'}
    \infer1[ (E - MatObsAssignCol) ]{ (M[N][N_2] \leftarrow P)[V][W] \rightarrow (M[N'][N_2] \leftarrow P)[V][W]}
\end{prooftree}
}
\end{center}

\begin{center}
\scalebox{1}{
\begin{prooftree}
    \hypo[]{ N \rightarrow N'}
    \infer1[ (E - MatObsAssignRow) ]{ (M[V][N] \leftarrow P)[V][W] \rightarrow (M[V][N'] \leftarrow P)[V][W])}
\end{prooftree}
}
\end{center}

\begin{center}
\scalebox{1}{
\begin{prooftree}
    \hypo[]{ P \rightarrow P'}
    \infer1[ (E - MatObsAssignValue) ]{ (M[V_2][W_2] \leftarrow P)[V_1][W_1] \rightarrow (M[V_2][W_2] \leftarrow P')[V_1][V_2]}
\end{prooftree}
}
\end{center}

\begin{center}
\scalebox{1}{
\begin{prooftree}
    \hypo[]{ V_2 \equiv V_1 \land W_2 \equiv W_1}
    \infer1[ (E - MatObsAssignGetValue) ]{ (M[V_2][W_2] \leftarrow Y)[V_1][W_1] \rightarrow Y}
\end{prooftree}
}
\end{center}

\begin{center}
\scalebox{1}{
\begin{prooftree}
    \hypo[]{ M \rightarrow M'}
    \infer1[ (E - MatObsRed) ]{ M \leftarrow Y)[V_1][W_1] \rightarrow Y}
\end{prooftree}
}
\end{center}

\begin{center}
\scalebox{1}{
\begin{prooftree}
    \hypo[]{ N \rightarrow N' }
    \infer1[ (E - MatConst) ]{ MAT(N)[V][W] \rightarrow MAT(N')[V][W]}
\end{prooftree}
}
\end{center}

\begin{center}
\scalebox{1}{
\begin{prooftree}
    \hypo[]{ }
    \infer1[ (E - MatConstGetValue) ]{ MAT(V)[V_1][W_1] \rightarrow V}
\end{prooftree}
}
\end{center}

\begin{center}
\scalebox{1}{
\begin{prooftree}
    \hypo[]{ }
    \infer1[ (E - MatObsMap) ]{ (Map(M[V_2][W_2], F))[V][W] \rightarrow Map(M, F)[V][W]}
\end{prooftree}
}
\end{center}


*No se me ocurrian mejores nombres*

\begin{center}
\scalebox{1}{
\begin{prooftree}
    \hypo[]{ M \rightarrow M'  }
    \infer1[ (E - MapReduceMat) ]{ (Map(M, F))[V][W] \rightarrow Map(M', F)[V][W]}
\end{prooftree}
}
\end{center}


\begin{center}
\scalebox{1}{
\begin{prooftree}
    \hypo[]{ V_2 \not\equiv V \lor  W \not\equiv W_2  }
    \infer1[ (E - MapMatAssignNotIndex) ]{ (Map(M[V_2][W_2] \leftarrow Y, F))[V][W] \rightarrow Map(M, F)[V][W]}
\end{prooftree}
}
\end{center}

\begin{center}
\scalebox{1}{
\begin{prooftree}
    \hypo[]{ V_2 \equiv V \land W \equiv W_2  }
    \infer1[ (E - MapMatAssign) ]{ (Map(M[V_2][W_2] \leftarrow Y, F))[V][W] \rightarrow F Y}
\end{prooftree}
}
\end{center}

\begin{center}
\scalebox{1}{
\begin{prooftree}
    \hypo[]{ V_2 \equiv V \land W \equiv W_2  }
    \infer1[ (E - MatConstValueMap) ]{ (Map(MAT(Z), F))[V][W] \rightarrow F Z}
\end{prooftree}
}
\end{center}




\inciso{e)} I) 

$$
 (MAT(\labsApp{y}{Bool}{y})[\underline{1}][0] \leftarrow (\labsApp{x}{Bool}{true}))[0][\underline{1}] \rightarrow_{(E - MatObsAssignGetValue)} 
  \labsApp{x}{Bool}{true} 
$$

\hfill\\

\noindent II) $$ ((\labsApp{x}{Bool}{MAT(x)}[0][0] \leftarrow False) true)[0][pred(\underline{1})] \rightarrow_{(E - MatObsRow), (E - PredSucc)}$$
$$ ((\labsApp{x}{Bool}{MAT(x)}[0][0] \leftarrow False) true)[0][0] \rightarrow_{ (E - MatObsRed), (E - AppsAbs / \beta)} $$
$$ (MAT(True)[0][0] \leftarrow False)[0][0] \rightarrow_{ (E - ObsAssignGetValue)} False $$

\hfill\\

\noindent III) $$ map(MAT(0)[0][0] \leftarrow \underline{1}, \labsApp{x}{Nat}{isZero(x)})[0][0] \rightarrow_{(E - MapMatAssign)}$$
$$ (\labsApp{x}{Nat}{isZero(x)}) \, \underline{1} \rightarrow_{(E - AppAbs / \beta)} isZero(\underline{1}) \rightarrow_{(E - IsZeroSucc)} False $$
\end{document}
