\documentclass[leqno, 12pt, twoside, letterpaper]{book}
\usepackage{silence}
\usepackage{../packages/notas}
\usepackage{../packages/lambda}
\usepackage{color}
\usepackage{hyperref}
\hypersetup{
    colorlinks=true,
    linkcolor=blue,
    urlcolor=red,
    linktoc=all,
    hypertexnames=false,
    linktocpage=true,
}
\usepackage{graphicx}

\usepackage{pgfplots}
\pgfmathdeclarefunction{poiss}{1}{%
  \pgfmathparse{(#1^x)*exp(-#1)/(x!)}%
}
\usepackage{float}

%\setlength\parindent{0pt}
\date{\vspace{-5ex}}

\makeindex

\begin{document}

\bigskip

\centerline {\Large Trabajo práctico 2}
\centerline {Leandro Javier Raffo LU:945/12 \quad  email: leandrojavr@gmail.com}

\section*{Ejercicio 8}
\noindent Nos piden la probabilidad de obtener k éxitos antes de r fracasos en una
sucesión de ensayos Bernoulli.\\

\noindent En una sucesión de ensayos Bernoulli la probabilidad de cada intento
es independiente de la anterior y tenemos como probabilidad de éxito $p$ y de
fracaso $(1 -p)$. Además nos piden que los éxitos ocurran (en total) antes que
los fracasos, esto es que entre exitos y fracasos en los primeros $k + r - 1$
intento tengamos $k$ éxitos. \\

\noindent Esto lo podemos conseguir todas las posibles
permutaciones de los órdenes de intentos y fracasos. La permutación es $(k + r -
1)!$ pero como no distinguimos entre exitos y fracasos tenemos que sacar las
permutaciones repetidas diviendo por $k!$ y $(r - 1)!$. Es decir que las
permutaciones son

$$ \dfrac{(k + r - 1)!}{k! r!} $$

\noindent Esto es solo la permutación, falta agregar la probabilidad de los intentos, que
como dijimos son independientes da

$$ \dfrac{(k + r - 1)!}{k! r!} p^k (1-p)^{r-1}$$


\section*{Ejercicio 11}

Tenemos $X \sim Poisson(\lambda = 2)$ que representa la cantidad de cajones
vendidos(demandados) por semana, además el minorista stockea 4 cajones por
semana.

\hfill\\
a) Queremos ver la probabilidad de vender los 4 cajones durante la semana, esto
no es nada más que

$$P(X = 4) = p_X(4) = \dfrac{e^{-2} 2^4}{4!} = 0.09$$

\hfill\\
b) La probabilidad entre semanas es independiente una de la otra asi que vender
todo el stock en al menos dos semanas es

$$P(X = 4)^2 P(X \leq 3)^2$$

Ahora como

$$P(X \leq 3) = p_X(0) + p_X(1) + p_X(2) + p_X(3)$$
$$P(X \leq 3) = \sum_{i = 0}^{3} \dfrac{e^{-2} 2^i}{i!}$$
$$P(X \leq 3) =  e^{-2}(1 + 2 + 4/2 + 8/6) = 0.857$$

Con lo cual

$$P(X = 4)^2 P(X \leq 3)^2 = (0.09)^2 (0.857)^2 = 5.95 x 10^{-3}$$

\hfill\\
c) Para ser capaz de incumplir un pedido significa que nos demandaron más de 4
cajones

$$P(X > 4) = 1 - P(X \leq 4)$$

Usando parte del punto b) tenemos

$$P(X \leq 3) =  e^{-2}(1 + 2 + 4/2 + 8/6 + 16/24) = 0.947$$


\hfill\\
d) Para calcular esta probabilidad nos preguntamos cuantos cajones seria
necesario vender para que la probabilidad sea menor que $0.1$, osea que si
stockeamos $T$ cajones la probabilidad de venderlos todos sea muy chica, puesto
en funcion de la probabilidad seria:

$$P(X > T) \leq 0.1$$

Pero esto es lo mismo que preguntar el $T$ para que

$$1 - P(X \leq T) \geq 0.99$$

Usando que

$$P(X \leq T) = \sum_{i = 0}^{T} p_X(i) = \sum_{i = 0}^T \dfrac{e^{-2}
2^i}{i!}$$

Lo que queremos es que la última suma sea mayor que $0.99$ (Recordemos que
converge a $1$).

Sacando de factor común el $e^{-2}$ tenemos que ver que tan cerca está la suma
de $e^2$, si tomamos $T = 7$ tenemos

$$e^2 - (1 + 2 + 4/2 + 8/6 + 16/24 + 32/120 + 64/720 + 128/5040) = e^2 - 7.38
\sim 0.008$$

Osea que si tomamos $T = 7$ como la cantidad mínima de cajones la probabilidad
de poder cumpliar todos los pedidos es mayor que $0.99$ ($\sim 0.992$)

\hfill\\
e) Acá la verdad que no entiendo que me esta pidiendo, el número de cajones
vendidos por semana tiene una distribución de Poisson con $\lambda = 2$ que es
la del siguiente gráfico (sigue hasta el infinito pero los valores están muy
cerca de 0)

\begin{figure}[H]
    \centering
\begin{tikzpicture}
\begin{axis}[every axis plot post/.append style={
  samples at = {0,...,20},
  axis x line*=bottom,
  axis y line*=left,
  enlargelimits=upper}]
  \addplot +[ycomb, mark=square*, color=red, mark options={fill=red}] {poiss(2)};
\end{axis}
\end{tikzpicture}
\end{figure}

\end{document}
