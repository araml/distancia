\documentclass[leqno, 12pt, twoside, letterpaper]{book}
\usepackage{../packages/notas}
\usepackage{../packages/lambda}
\usepackage{color}
\usepackage{hyperref}
\usepackage{qtree}
\usepackage{minted}

\hypersetup{
    colorlinks=true,
    linkcolor=blue,
    urlcolor=red,
    linktoc=all,
    hypertexnames=false,
    linktocpage=true,
}
\usepackage{graphicx}

%\setlength\parindent{0pt}
\date{\vspace{-5ex}}

\def\checkmark{\tikz\fill[scale=0.4](0,.35) -- (.25,0) -- (1,.7) -- (.25,.15) -- cycle;}

\title{Autoevaluación 2do cuatrimestre 2020}
\makeindex

\begin{document}
\maketitle
\tableofcontents

\newpage

\exercise{Ejercicio 1 - Programación Funcional}\\


\inciso{a)}
\begin{minted}{Haskell}
ejGrafo :: Grafo
ejGrafo = G [1..9] (\x -> if mod x 2 == 0 then [1..9] else [])
\end{minted}

\inciso{b)}
\begin{minted}{Haskell}
nodoReflex :: Grafo -> Nodo -> Bool
nodoReflex G ns f n = n elem (f n)
\end{minted}

\inciso{e)}
\begin{minted}{Haskell}

ocurrencias :: Int -> [Int] -> Int
ocurrencias n ls = len (filter (\x -> x != n) ls)

sinNodosRepetidos :: [Int] -> Bool
sinNodosRepetidos ls = foldr (\x r ->
                              if (ocurrencias x ls) != 1 then
                                  false
                              else
                                true && r) true ls

hamiltoniano :: Grafo -> Bool
hamiltoniano G ns f = len (filter sinNodosRepetidos
                                  (caminosDeLong  (g n f) (len ns - 1))
\end{minted}

\inciso{f)}
\begin{minted}{Haskell}
tieneCiclo :: Grafo -> Bool
tieneCiclo G ns f = length (caminosDeLong (G ns f) (len ns + 1)) > 0
\end{minted}

\newpage

\exercise{Ejercicio 2 - Cálculo Lambda Tipado}\\

\inciso{a)}

\[
\begin{prooftree}
    \hypo{\juicio N : \sigma}
    \infer1[ (T - MatInit) ]{ \juicio MAT(N) : MatInf_{\sigma}}
\end{prooftree}
\]

\hfill

\[
\begin{prooftree}
    \hypo{\juicio M : MatInf_{\sigma} \quad  \juicio N, O: Nat}
    \infer1[ (T - MatObs) ]{ \juicio M[N][P] : \sigma}
\end{prooftree}
\]

\hfill

\[
\begin{prooftree}
    \hypo{\juicio M : MatInf_{\sigma} \quad \juicio P : \sigma \quad \juicio N, O: Nat}
    \infer1[ (T - MatIns) ]{ \juicio M[N][P] \leftarrow P : MatInf_{\sigma} }
\end{prooftree}
\]


\hfill

\[
\begin{prooftree}
    \hypo{\juicio M : MatInf_{\sigma} \quad \juicio N : \sigma \rightarrow \tau}
    \infer1[ (T - MatMap) ]{ \juicio map(M, N) : MatInf_{\tau}}
\end{prooftree}
\]

\inciso{b)} Tomando $\Gamma = \pset{i : Nat, m : MatInf_{Nat}}$

\begin{center}
\scalebox{0.9}{
    \begin{prooftree}
        \infer0[]{\juicio (m[i][i] \leftarrow 0) : MatInf_{\sigma} \quad \juicio \labsApp{x}{Nat}{isZero(x))} : \sigma \rightarrow Bool}
        \infer1[ (T - Map) ]{ \juicio map((m[i][i] \leftarrow 0), \labsApp{x}{Nat}{isZero(x))} : MatInf_{Bool}}

    \end{prooftree}
}
\end{center}

\end{document}
