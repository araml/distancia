\chapter{Practica Nº 5 - Programación Orientada a Objetos}

\exercise{Ejercicio 13}
\begin{solucion*}
$$
    \object{true}{if = \varsigma(v) \lambda(x)(\lambda(y) x),\,
                  ifnot = \varsigma(v) \lambda(x)(\lambda(y) y),\,
                  not = \varsigma(v) false}
$$

$$
    \object{false}{if = \varsigma(v) \lambda(x)(\lambda(y) y),\,
                   ifnot = \varsigma(v) \lambda(x)(\lambda(y) x),\,
                   not = \varsigma(v) true}
$$

\end{solucion*}


\exercise{Ejercicio 14}

\begin{solucion*}

\inciso{a)}
\begin{align*}
\object{origen}
    {&x = 0, y = 0, \\
    &mv\_x = \varsigma(s) \lambda(dx) s.x = s.x + dx, \\
    &mv\_y = \varsigma(s) \lambda(dy) s.y = s.y + dy, \\
    &mv = \varsigma(s) \lambda(dx)(\lambda(dy) s.mv\_y \, dy).mv\_x \, dx)}
\end{align*}

\inciso{b)}
\begin{align*}
\object{punto}
    {&new = \varsigma(z) \bracket{ \begin{aligned}[t]
                               &x = 0, y = 0,\\
                               &\varsigma(s)z.mv\_x(s),\\
                               &\varsigma(s)z.mv\_y(s),\\
                               &\varsigma(s)z.mv(s)}
        \end{aligned}\\
     &mv\_x = \lambda(z)\lambda(dx) z.x = z.x + dx, \\
     &mv\_y = \lambda(z)\lambda(dy) z.y = z.y + dy, \\
     &mv = \lambda(z)\lambda(dx)(\lambda(dy) z.mv\_y \, dy).mv\_x \, dx }
\end{align*}

\inciso{c)}

\[
    \begin{prooftree}
        \infer0[\bracket{OBJ}]{p \rightarrow p}
        \hypo{\bracket{x = 0, y = 0, \varsigma(s)z.mv\_x(s),
                           \varsigma(s)z.mv\_y(s),
                           \varsigma(s)z.mv(s)}\pset{z \leftarrow p}}
        \infer2[\bracket{SEL}]{\underbrace{punto}_{p}.new \rightarrow
                            \bracket{x = 0, y = 0, \varsigma(s)p.mv\_x(s),
                            \varsigma(s)p.mv\_y(s),
                            \varsigma(s)p.mv(s)}}
    \end{prooftree}
\]

\end{solucion*}

\exercise{Ejercicio 15}

\begin{solucion*}
\begin{align*}
\object{vacio}{&hayElementos = false, \\
               &pertenece = \varsigma(s) \lambda(x) false, \\
               &sacar = \varsigma(s) \lambda(x), \\
               &agregar = \varsigma(s) \lambda(e) ((s.hayElementos
               \leftleftharpoons true).pertenece \\
               &\leftleftharpoons \lambda(x) \ITE{x = e}{true}{s.pertenece(x)}).sacar \\
               &\leftleftharpoons \lambda(x) \ITE{x = e}{s.sacar(x)}{s.sacar(x).agregar(e)}}
\end{align*}
\end{solucion*}


\exercise{Ejercicio 16}

\begin{solucion*}

\noindent Nota: abrevio pC para plantaClass.

\[
    \begin{prooftree}
        \hypo{\overbrace{pC.new \rightarrow p'}^{(1)}}
        \hypo{p' = [] }
        \hypo{}
        \hypo{}
        \infer4[\bracket{SEL}]{\underbrace{plantaClass.new}_{p}.crecer \rightarrow
                            \bracket{}}
    \end{prooftree}
\]


\begin{equation}\tag{1}
    \scalebox{0.9}{
    \begin{prooftree}
        \infer0[\bracket{OBJ}]{pC \rightarrow pC}
        \hypo{pC = \overbrace{\bracket{altura = c.altura, crecer = \varsigma(s)c.crecer(s)
        }\pset{c \leftarrow pC}}^{(2)} }
        \infer2[\bracket{SEL}]{}
    \end{prooftree}
}
\end{equation}

\begin{equation}\tag{2}
    \scalebox{0.7}{
    \begin{prooftree}
        \hypo{[altura = pC.altura, crecer = \varsigma(s)pC.crecer(s)]
        \rightarrow [altura = 10, crecer = \varsigma(t)\varsigma(c) t.altura :=
        t.altura + 10] }
        \infer1[]{}
    \end{prooftree}
}
\end{equation}

\noindent Reemplazando al principio tenemos

\[
    \begin{prooftree}
        \hypo{\overbrace{pC.new \rightarrow p'}^{(1)}}
        \hypo{p' =  [altura = 10, crecer = \varsigma(t)\varsigma(c) t.altura :=
        t.altura + 10] }
        \hypo{}
        \hypo{}
        \infer4[\bracket{SEL}]{\underbrace{plantaClass.new}_{p}.crecer \rightarrow
                            \bracket{}}
    \end{prooftree}
\]


\inciso{d)}

    \[ malezaClass = plantaClass.crecer \leftleftarrows \varsigma(s)(\lambda(t)t.altura := (t.altura * 2))  \]


\end{solucion*}


