\documentclass[leqno, 12pt, twoside, letterpaper]{book}
\usepackage{../packages/notas}
\usepackage{../packages/lambda}
\usepackage{color}
\usepackage{hyperref}
\usepackage{qtree}

\hypersetup{
    colorlinks=true,
    linkcolor=blue,
    urlcolor=red,
    linktoc=all,
    hypertexnames=false,
    linktocpage=true,
}
\usepackage{graphicx}

%\setlength\parindent{0pt}
\date{\vspace{-5ex}}

\def\checkmark{\tikz\fill[scale=0.4](0,.35) -- (.25,0) -- (1,.7) -- (.25,.15) -- cycle;}

\title{Solucion a los ejercicios de Paradigmas de Lenguajes de Programación}

\begin{document}

\exercise{1)} \\

\inciso{a)}

Podemos pensar al intervalo como un caso particular de una lista de enteros, ya
que este modela una tira de enteros seguida. Por eso si tenemos una lista de
algo a lo cual Int subtipe podemos reemplazar el tipo de la lista por el de un
intervalo.

\[
\begin{prooftree}
    \hypo{Int <: \sigma}
    \infer1[ (S - SubIntervalo) ]{ intervalo <: [\sigma]}
\end{prooftree}
\qquad
\begin{prooftree}
    \hypo{Int <: \sigma}
    \infer1[ (S - SubIntervalo) ]{ it_{intervalo} <: \it_{[\sigma]}}
\end{prooftree}
\]


\hfill\\
\inciso{b)}

\inciso{II)} No va a tipar pues la función espera un intervalo, es decir vamos a
tener que tipar la lista como intervalo y no tenemos ninguna regla para cambiar
esto por lista y que tipe correctamente.


\hfill\\
\inciso{I)}

\begin{center}
    \scalebox{0.70}{
        \begin{prooftree}
            \hypo{ \checkmark}
            \infer1[\TVar]{\juicioSet{i : It_{[Float]}} i : It_{[Float]}}
            \infer1{\juicioSet{i : It_{[Float]}} proximo(i) : Float }
            \infer1[\TAbs]{\juicioVacio \labsApp{i}{It_{[Float]}}{proximo(i)} :
            It_{[Float]} \to Float }
            \hypo{\juicioVacio -1 : Float}
            \hypo{... \checkmark}
            \infer1{\juicioVacio [0, \, Succ(0)] : Intervalo }
            \hypo{ Int <: Float }
            \infer1[ (S-SubInt) ] {Intervalo <: [Float] }
            \infer2[ (S - Subs) ]{\juicioVacio [0, Succ(0)] : [Float]}
            \infer2[ (T - ListaHead)]{\juicioVacio -1 :: [0, Succ(0)] : [Float] }
            \infer1[(T - ItLista)]{\juicioVacio iterar(-1 :: [0, Succ(0)]) : It_{[Float]} }
            \infer2[\TApp]{\juicioVacio (\labsApp{i}{It_{[Float]}}{proximo(i)})\,\, iterar(-1 ::
            [0, Succ(0)]) : Float}
        \end{prooftree}
    }
\end{center}

\hfill\\

\exercise{2)}
\inciso{b)}
\hfill\\

\begin{align}
    iS = [&next = \varsigma(s)[\varsigma(z)next = s.next(z),\, s.val = s.val + 1], \\
          &val = 0 ]
\end{align}

\end{document}
